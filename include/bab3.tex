
\chapter{METODE PENENILITIAN}


\section{Tahapan Penelitian}
Tahapan dalam penelitian ini dapat dilihat pada gambar \ref{fig:penelitian-flowchart}.
\begin{afigure}
    \includegraphics[width=13cm, center]{images/penelitian-flowchart.jpg}
    \caption{Flowchart tahapan penelitian.}
    \label{fig:penelitian-flowchart}
\end{afigure}
% tambahkan penjelasan lagi


\section{Waktu dan Lokasi Penelitian}
Penelitian ini dilaksanakan dari bulan Juni 2020 sampai dengan bulan Agustus 2020. Lokasi penelitian dilakukan di Laboratorium Rekayasa Perangkat Lunak Fakultas Matematika dan Ilmu Pengetahuan Alam, Universitas Hasanuddin Makassar.

\section{Rancangan Sistem}
Pada penelitian ini akan dibangun suatu sistem untuk mengimplementasikan filter spasial linear pada FPGA, dapat dilihat pada gambar \ref{fig:rancangan-sistem}.
\begin{afigure}
    \includegraphics[width=0.85\textwidth, center]{images/rancangan-sistem.jpg}
    \caption{Rancangan sistem.}
    \label{fig:rancangan-sistem}
\end{afigure}

Video \textit{stream} dari \textit{source} disalurkan melalui port HDMI Input pada FPGA Board, kemudian video \textit{stream} tersebut akan diolah dengan menerapkan filter spasial linear pada setiap framenya. Setiap frame yang telah diterapkan filter spasial akan dialirkan ke monitor untuk kemudian ditampilkan. Selanjutnya dilakukan analisis kinerja pada FPGA. FPGA board yang digunakan dalam penelitian ini dapat diakses dengan \textit{ssh} pada port 22 atau dengan Jupyter Notebook melalui browser.

\pagebreak

\section{Instrumen Penelitian}
Instrumen penelitian ini yaitu:
\begin{enumerate}[topsep=0pt,itemsep=0pt,partopsep=0pt, parsep=0pt]
    \item Kebutuhan perangkat lunak:
    \begin{enumerate}[topsep=0pt,itemsep=0pt,partopsep=0pt, parsep=0pt, label={\alph*.}]
        \item Ubuntu 18, sebagai OS pada FPGA Board.
        \item Python 3.6, dengan modul OpenCV, Numpy, Pynq 5.2, Xilinx xfOpenCV dan beberapa modul pendukung lainnya.
        \item Jupyter Notebook pada FPGA Board. 
        \item Web Browser untuk mengakses Jupyter Notebook pada FPGA Board.
    \end{enumerate}
    \item Kebutuhan perangkat keras:
    \begin{enumerate}[topsep=0pt,itemsep=0pt,partopsep=0pt, parsep=0pt, label={\alph*.}]
        \item FPGA Board Xilinx PYNQ-Z2.
        \item Micro SD Card 16Gb, sebagai media penyimpanan OS pada FPGA Board.
        \item Monitor Eksternal, untuk menampilkan hasil penerapan filter spasial pada FPGA Board.
        \item Laptop Lenovo Ideapad 320 (sebagai \textit{source} video \textit{stream}).
    \end{enumerate}
\end{enumerate}
